\documentclass[12pt]{report}

\begin{document}
\title{Mediawiki}
\section{Was ist Mediawiki?}
MediaWiki ist eine freie Server-basierte Software, die unter der GNU General Public License (GPL) lizenziert ist. MediaWIKI wurde entworfen, um auf einer großen Server-Farm eine Website zu betreiben, die Millionen Seitenzugriffe pro Tag erhält. MediaWiki ist eine äußerst leistungsfähige, skalierbare Software und eine funktionsreiche Wiki-Implementierung, die PHP verwendet, um Daten zu verarbeiten und anzuzeigen, die in einer Datenbank wie MySQL gespeichert sind. Auf den einzelnen Webseiten wird MediaWikis Wikitext-Format verwendet, so dass Anwender ohne Kenntnisse von XHTML oder CSS sie einfach bearbeiten und gestalten können.Wenn ein Benutzer eine Bearbeitung auf einer Seite anlegt, schreibt MediaWiki es in die Datenbank, aber ohne die vorherigen Versionen der Seite zu löschen, so dass einfache Zurücksetzungen im Falle von Vandalismus oder Spam möglich sind. MediaWiki kann auch Bild-und Multimedia-Dateien verwalten, die im Dateisystem gespeichert werden. Für große Wikis mit vielen Benutzern, unterstützt MediaWiki Caching und kann leicht mit Squid-Proxy-Server-Software gekoppelt werden.
\section{Was ist ein Wiki?}
Ein Wiki ist Hypertextsystem für Webseiten, deren Inhalte von den Benutzern nicht nur gelesen, sondern auch online direkt im Webbrowser geändert werden können. Das Ziel ist häufig, Erfahrung und Wissen gemeinschaftlich zu sammeln (kollektive Intelligenz) und in für die Zielgruppe verständlicher Form zu dokumentieren. Die Autoren erarbeiten hierzu gemeinschaftlich Texte, die ggf. durch Fotos oder andere Medien ergänzt werden. Ermöglicht wird dies durch ein vereinfachtes Content-Management-System, die sogenannte Wiki-Software oder Wiki-Engine. Die bekannteste Anwendung von Wikis ist die Online-Enzyklopädie Wikipedia, welche die Wiki-Software MediaWiki einsetzt. Als wesentlicher Unterschied zu anderen Content-Management-Systemen (CMS) bietet Wiki-Software weniger Gestaltungsmöglichkeiten für Layout und Design der Webseiten. Primäre Funktion ist hingegen ein Bearbeitungsmodus für jede Wiki-Seite, der es auch einem Neuling erlaubt, ohne große Einarbeitung Text und Inhalt der Seite zu ändern. Im Unterschied zu den Content-Management-Systemen (CMS) mit ihren teils genau geregelten Arbeitsabläufen (engl. workflows) etwa in Redaktionssystemen, setzen Wikis auf die Philosophie des offenen Zugriffs: idealerweise kann jeder Nutzer jeden Eintrag lesen und bearbeiten. Wikis gelten als gegenüber einem klassischen CMS dann im Vorteil, wenn eine hohe Anzahl an Nutzern Informationen einstellt, so dass im Medium eine kritische Masse erreicht wird und es zu einem „Selbstläufer“ wird. Eine wesentliche Funktion der meisten Wiki-Produkte ist die Versionsverwaltung, die es den Nutzern im Fall von – durch den offenen Zugriff kaum vermeidlichen – Fehlern oder Vandalismus erlaubt, eine frühere Version einer Seite schnell wierherzustellen. Wie bei Hypertexten üblich, sind die einzelnen Seiten eines Wikis durch Querverweise (Hyperlinks) miteinander verbunden. Dabei dient der Titel einer Seite meist auch als Linkadresse. Links auf nichtexistente Seiten werden dann nicht als Fehler angezeigt, sondern es erscheint ein Formular, um die neue Seite anzulegen. Eine Vernetzung mit anderen populären Wiki-Diensten wird teils durch sog. InterWiki-Verweise ermöglicht. Die meisten Systeme sind als freie Software veröffentlicht, oft unter einer Version der gebräuchlichen GNU General Public License (GPL). Viele Wiki-Software Systeme sind modular aufgebaut und bieten eine eigene Programmierschnittstelle, welche dem Benutzer ermöglicht, eigene Erweiterungen zu schreiben, ohne den gesamten Quellcode zu kennen. Ein Wiki kann öffentlich zugänglich im World Wide Web verfügbar sein, in lokalen Netzwerken nur für eine bestimmte Nutzergruppe (z. B. als Intranet) eingesetzt werden oder auch auf einem einzelnen Rechner zur persönlichen Informationsorganisation verwendet werden, etwa in Form eines Desktop-Wikis. 
\section{Geschichte}
MediaWiki entstand aus einer Wiki-Engine, die der deutsche Biochemiker Magnus Manske für die Online-Enzyklopädie Wikipedia entwickelte, als sich die zuvor eingesetzte UseModWiki-Engine den Anforderungen nicht gewachsen zeigte. Am 25. Januar 2002 wurde die erste Version, damals Phase II genannt, erstmals eingesetzt. Nach einer hauptsächlich durch Lee Daniel Crocker geschriebenen Neufassung wurde im Juni 2002 eine verbesserte Version der offiziell immer noch namenlosen Software auf dem Wikipedia-Server installiert. Der heutige Name MediaWiki wurde erstmals im Juli 2003 von dem Entwickler Daniel Meyer auf einer Mailingliste vorgeschlagen. Das Logo der Software zeigt eine von eckigen Klammern umgebene Sonnenblume und stammt von Erik Möller nach einem Foto von Florence Nibart-Devouard. Es wurde 2003 bei einem Wikipedia-Wettbewerb für das MediaWiki-Projekt gewählt. In den Folgejahren entwickelte sich MediaWiki zu einem erfolgreichen Open-Source-Projekt, an dem im Jahr 2005 über 60 Programmierer und Helfer beteiligt waren. Neben Wikipedia und ihren Wikimedia-Schwesterprojekten setzen heute zahlreiche Organisationen, Unternehmen und Institutionen MediaWiki ein.
\section{Gruppen}
\begin{enumerate}
	\item Alle Benutzer:
	Jeder Benutzer – sowohl ein angemeldeter als auch ein „anonymer“ – darf Seiten anlegen und bearbeiten.
	\item Angemeldeter Benutzer (user):Angemeldete Benutzer können zusätzlich Seiten verschieben und Dateien (z. B. Bilder) hochladen.
	\item Administrator (sysop): Admins können Seiten schützen und geschützte Seiten bearbeiten, Seiten löschen und gelöschte Seiten wiederherstellen. Außerdem haben sie die Möglichkeit, andere Benutzer bzw. IPs zu sperren und solche Sperren wieder aufzuheben.
	\item Bürokrat (bureaucrat): Ein Bürokrat kann anderen Benutzern Bot-, Administrator- und Bürokraten rechte erteilen und entziehen.
	\item Bot (bot): Ein Benutzer mit dem Status „Bot“ darf mit Hilfe eines Programms oder eines Skripts stupide, langweilige und häufig auftretende Aufgaben erledigen.
\end{enumerate}

\end{document}
