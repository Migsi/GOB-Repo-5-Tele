\section{Bonita_BPM}

Bonita Open Solution ist eine open-source Software f�r die Planung und Gestaltung des Business Process Management (BPM). Die Software kann Gesch�ftsprozesse im PMBO Standart 2.0 grafisch darstellen. Dies erlaubt dem Benutzer den Prozess zu planen und anschlie�end direkt in einen Arbeitsablauf zu konvertieren. Dar�ber hinaus erm�glicht Bonita es dem Benutzer, die Prozesse mit anderen Standards und Technologien zu erg�nzen. BOS ist mehrsprachig gehalten und unterst�tzt Englisch, Franz�sisch, Spanisch sowie Deutsch.

\subsection{Hauptkomponenten}

Bonita hat drei Hauptkomponenten:
\begin{itemize}
\item Die Software kann Gesch�ftsprozesse im PMBO Standart 2.0 grafisch darstellen.
\item Die Business Process Management Engine ist eine JAVA API, die programmgesteuert mit Prozessen interagieren kann.
\item Das Bonita Portal. Das Portal wird von Endbenutzern verwendet, um ihre Projekte, oder Teilprojekte zu verwalten.
\end{itemize}

\subsection{Lizensierung}

Das Bonita Programm ist frei im Internet erh�ltlich, hatt allerdings die Copyleft-Lizens GPL. Das beteutet, dass jedes Projekt, das mit diesem Programm geschrieben wurde, unter GPL freigegeben werden muss. Somit ist es f�r gr��ere Firmen nicht gut brauchbar. 

\subsection{Verwendung und Herkunft}

Im Jahre 2001 w�rde Bonita in Frankreich entwickelt und ver�ffentlicht. DIe Hersteller Firma tr�gt den Namen Bonitasoft. Bonita kann dazu verwendet werden, um high-tech Arbeitsabl�ufe zu planen. Die Software wurde das letzte mal im Dezember 2014  gewartet.
