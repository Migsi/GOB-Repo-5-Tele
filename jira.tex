\subsection{Jira}
\subsubsection{Geschichte}
Jira wurde erstmals 2002 von der australischen Softwarefirma Atlassian Inc. auf den Markt gebracht und wurde in Java geschrieben. Der Name wurde vom ursprünglichen japanischen Namen für Godzilla, „Gojira“, abgeleitet. Die Entwickler wollten damit einen Bezug zu Bugzilla erhalten.
\subsubsection{Funktionsweise}
Jira ist eine webbasierte Anwendung zur Fehlerverwaltung, Problembehandlung und Projektmanagement. Sie basiert auf einer Java Servlet Plattform und läuft auf verschiedenen Datenbanken und Betriebssystemen. Jira wird vom Nutzer mit sogenannten Tickets gefüllt. Diese sind auf verschiedene Arten einsehbar, zum Beispiel Dashboards, Suchfilter, Statistiken oder per E-Mail. Ein Ticket besteht aus einer Projektzuordnung, einer Zusammenfassung, einem Typ, einer Priorität, einer oder mehreren Kompetenzzuordnungen und einem Inhalt. Des weiteren ist es möglich selbst Felder zu definieren und weitere Informationen durch Anhänge oder Kommentare hinzuzufügen. Diese Tickets können editiert werden oder den Status wechseln. Welche Statuswechsel dabei möglich sind definiert der anpassbare Workflow. Jede Änderung wird in einem Log festgehalten. Jira besitzt eine große Anzahl an Konfigurationsmöglichkeiten. So kann für jeden Anwendungszweck eine eigener Tickettyp mit eigenem Workflow, eigenem Status, einer oder mehreren Ansichten, eigenen Feldern und beliebigen Workflowübergängen erstellt werden. Zusätzlich ist es möglich durch sogenannte „Schemes“ die Zugriffe, das Verhalten von Feldern, die Sichtbarkeit der Tickets und mehr für jedes Projekt individuell festzulegen. Die flexible Architektur von Jira ermöglicht es dem Benutzer Erweiterungen für Jira zu entwickeln und diese über den Atlassian Marketplace zur Verfügung zu stellen.
\subsubsection{Nutzer und Lizenzen}
Jira wird weltweit  von ca. 14500 Kunden genutzt. Auf der Liste der Nutzer stehen unter anderem IBM, Microsoft, Nokia, Electronic Arts, BMW und Audi, aber auch Institutionen wie das Europäische Parlament und das CERN. Des weiteren wird Jira von großen Universitäten wie Harvard oder Stanford genutzt. Obwohl Jira ein kommerzielles Produkt ist, gibt es kostenlose Lizenzen für Open-Source-Projekte, gemeinnützige Einrichtungen, wohltätige Organisationen oder Einzelpersonen. So wird Jira in zahlreichen Apache-Projekten und von den Entwicklern von ReactOS genutzt. Beim Kauf einer kommerziellen Lizenz werden meist zusätzliche Serviceleistungen wie Installation, Wartung und Hosting angeboten.
