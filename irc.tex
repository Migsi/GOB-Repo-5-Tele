% Please enable Umlaute.

\subsection{Internet Relay Chat}
Als \textbf{Internet Relay Chat} (im Folgenden IRC) bezeichnet man ein rein text-basiertes Chat-System. Es ermöglicht Gespräche zwischen einer beliebigen Anzahl von Teilnehmern, die in sogenannte Kanäle (engl. Channels) eingeteilt werden. Neue Kanäle können meist von jedem Teilnehmer eröffnet werden und ein Teilnehmer kann auch an mehreren Kanälen parallel teilnehmen. Zur Teilnahme an einem Gespräch wird ein IRC-Client benötigt. 
\\
Zur Vermittlung dient ein IRC-Netzwerk, welches aus miteinander verbundenen Server (den Relais-Stationen) besteht. 
IRC basiert auf einer von \textit{BITNET} übernommenen Topologie, welche vorschreibt, dass zwischen zwei beliebigen Teilnehmern immer nur genau ein Kommunikationspfad besteht. Dies erlaubte es IRC trotz beschränkter Leitungskapazitäten sehr große Chat-Netzwerke zu  beherbergen.
\\
Die größten IRC-Netzwerke bestehen aus mehreren Dutzend Servern, welche gleichzeitig über 100.000 Benutzer verbinden, die in mehr als 10.000 Kanälen Gespräche führen können.
\\
Beim ursprünglichen IRC kommt ein auf IP/TCP basierendes, textorientiertes Protokoll zum Einsatz.
Nachrichten im IRC haben folgende Eigenschaften:
\begin{itemize}
\item Sämtliche Kommunikation zwischen Client und Server wird über Nachrichten in Befehlsform mit einer Maximallänge von 512 Zeichen abgewickelt.
\item Eine Nachricht besteht aus einem Absender, einem Befehl und zusätzlichen Befehlsparametern.
\item Server tauschen nur Nachrichten mit Absenderabgabe aus.
\item Als Antwort auf eie Nachricht von einem CLient kann ein Server eine Antwort-Nachricht schicken, die einen \textit{Reply-Code} hat. 
\end{itemize}
IRC kann sowohl unverschlüsselt, als auch über eine SSL/TLS-verschlüsselte Verbindung benutzt werden. Clientseitig ist es oftmals möglich Nachrichten zu verschlüsseln. Eine Möglichkeit zur Verschlüsselung bietet \textbf{FiSH}, welches mit Hilfe eines symmetrischen Kryptosystems einen Kanal verschlüsseln kann. Weiterhin können auch Gespräche zwischen zwei Nutzern durch ein Asymmetrisches Kryptosystem, auf Basis eines Diffie-Hellmann-Schlüsselaustauschs abgesichert werden.
\\
Es gibt tausende IRC-Netzwerke die zur Zeit aktiv sind. Die größten IRC-Netzwerke waren zum Höhepunkt der IRC-Nutzung:
\begin{itemize}
\item EFnet
\item IRCnet
\item Undernet
\item DALnet
\end{itemize}
Im März 2015 waren die größten Netzwerke folgende:
\begin{itemize}
\item freenode - 99k Users
\item IRCNet - 44k Users
\item QuakeNet - 36k Users
\item EFnet - 26k Users
\item Undernet - 25k Users
\item rizon - 25k Users
\end{itemize}
URL Schemas für IRC sind wie folgt aufgebaut:
\begin{quote}
\verb|irc://<host>[:<port>]/[<channel>[?<channel_keyword]]|
\end{quote}
Moderne IRC-Systeme unterstützen eine Vielzahl an Features, die im originalen IRC-Protokoll nicht vorgesehen waren.
\begin{itemize}
\item Services die von einem Bot bereitgestellt werden, wie zum Beispiel Nickname-Registrierung.
\item Proxy-Erkennung, um unsichere Proxy-Verbindungen zu erkennen und zu blocken.
\item Zusätzliche Kommandos, welche mit den Services Hand in Hand gehen.
\item Verschlüsselung. Zwischen Client und Server wird meist SSL genutzt, während von Client zu CLient meist Secure DCC zum Einsatz kommt.
\end{itemize}
