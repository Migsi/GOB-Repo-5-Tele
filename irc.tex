% Please enable Umlaute.

\section{Internet Relay Chat}
\subsection{Einführung}
Als \textbf{Internet Relay Chat} (im Folgenden IRC) bezeichnet man ein rein text-basiertes Chat-System. Es ermöglicht Gespr"ache zwischen einer beliebigen Anzahl von Teilnehmern, die in sogenannte Kan"ale (engl. Channels) eingeteilt werden. Neue Kan"ale k"onnen meist von jedem Teilnehmer er"offnet werden und ein Teilnehmer kann auch an mehreren Kan"alen parallel teilnehmen. Zur Teilnahme an einem Gespr"ach wird ein IRC-Client ben"otigt. 

Zur Vermittlung dient ein IRC-Netzwerk, welches aus miteinander verbundenen Server (den Relais-Stationen) besteht. 
IRC basiert auf einer von \textit{BITNET} "ubernommenen Topologie, welche vorschreibt, dass zwischen zwei beliebigen Teilnehmern immer nur genau ein Kommunikationspfad besteht. Dies erlaubte es IRC trotz beschr"ankter Leitungskapazit"aten sehr große Chat-Netzwerke zu beherbergen.

Die größten IRC-Netzwerke bestehen aus mehreren Dutzend Servern, welche gleichzeitig über 100.000 Benutzer verbinden, die in mehr als 10.000 Kan"alen Gespr"ache f"uhren k"onnen.

\subsection{Funktionsweise}
Beim urspr"unglichen IRC kommt ein auf IP/TCP basierendes, textorientiertes Protokoll zum Einsatz.
Nachrichten im IRC haben folgende Eigenschaften:
\begin{itemize}
\item S"amtliche Kommunikation zwischen Client und Server wird "uber Nachrichten in Befehlsform mit einer Maximall"ange von 512 Zeichen abgewickelt.
\item Eine Nachricht besteht aus einem Absender, einem Befehl und zus"atzlichen Befehlsparametern.
\item Server tauschen nur Nachrichten mit Absenderabgabe aus.
\item Als Antwort auf eine Nachricht von einem Client kann ein Server eine Antwort-Nachricht schicken, die einen \textit{Reply-Code} hat. 
\end{itemize}
IRC kann sowohl unverschl"usselt, als auch "uber eine SSL/TLS-verschl"usselte Verbindung benutzt werden. Clientseitig ist es oftmals m"oglich Nachrichten zu verschl"usseln. Eine M"oglichkeit zur Verschl"usselung bietet \textbf{FiSH}, welches mit Hilfe eines symmetrischen Kryptosystems einen Kanal verschl"usseln kann. Weiterhin können auch Gespr"ache zwischen zwei Nutzern durch ein Asymmetrisches Kryptosystem, auf Basis eines Diffie-Hellmann-Schl"usselaustauschs abgesichert werden.

URL Schemas für IRC sind wie folgt aufgebaut:
\begin{quote}
\verb|irc://<host>[:<port>]/[<channel>[?<channel_keyword]]|
\end{quote}
Moderne IRC-Systeme unterst"utzen eine Vielzahl an Features, die im originalen IRC-Protokoll nicht vorgesehen waren.
\begin{itemize}
\item Services die von einem Bot bereitgestellt werden, wie zum Beispiel Nickname-Registrierung.
\item Proxy-Erkennung, um unsichere Proxy-Verbindungen zu erkennen und zu blocken.
\item Zus"atzliche Kommandos, welche mit den Services Hand in Hand gehen.
\item Verschl"usselung. Zwischen Client und Server wird meist SSL genutzt, während von Client zu CLient meist Secure DCC zum Einsatz kommt.
\end{itemize}
\subsection{Aktive Netzwerke}
Es gibt tausende IRC-Netzwerke die zur Zeit aktiv sind. Die größten IRC-Netzwerke waren zum H"ohepunkt der IRC-Nutzung:
\begin{itemize}
\item EFnet
\item IRCnet
\item Undernet
\item DALnet
\end{itemize}
Im März 2015 waren die gr"oßten Netzwerke folgende:
\begin{itemize}
\item freenode - 99k Users
\item IRCNet - 44k Users
\item QuakeNet - 36k Users
\item EFnet - 26k Users
\item Undernet - 25k Users
\item rizon - 25k Users
\end{itemize}

