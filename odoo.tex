\section{Odoo}

\subsection{Allgemeine Daten}
Odoo ist ein offenes ERP-System. ERP bedeutet Enterprise Resource Planning und wird dazu verwendet Ressourcen in einem Unternehmen bedarfsgerecht und p"unktlich zu planen und zu steuern. Ressourcen k"onnen z.B. Material welches f"ur die Produktion ben"otigt wird, Arbeitskr"afte, Kapital oder IT-Systeme sein. ERP-Systeme sind eine Vielzahl von Softwares, welche zur Unterst"utzung des Ressourcenmanagements dienen. Die Hauptfunktionsbereiche sind:
\begin{itemize}
	\item Controlling
	\item Produktion bzw. Produktionsplanung und -steuerung
	\item Finanz- und Rechnungswesen
	\item Materialwirtschaft (Beschaffung, Lagerhaltung, Disposition)
	\item Verkauf und Marketing
	\item Personalwirtschaft
	\item Bedarfsermittlung
	\item Forschung und Entwicklung
	\item Produktdatenmanagement
	\item Stammdatenverwaltung
	\item St"uckliste
	\item Dokumentenmanagement
\end{itemize}


\subsection{Ersteller}
Odoo ist programmiert, supportet und organisiert von Odoo S.A. Das Unternehmen war fr"uher bekannt unter dem Namen OpenERP S.A. und wurde 2002 in Belgien von Fabien Pinckaers, der zurzeit CEO ist, gegr"undet. In 6 B"uros (Belgien, San Francisco, New York, Luxemburg, Indien und Hongkong) sind momentan 250 Angestellte besch"aftigt. Das Unternehmen weist in 120 L"andern Aktivit"at auf und besitzt ein Partnernetzwerk mit mehr als 550 offiziellen Partnern. Zudem wird Odoo als open-source Projekt von einer Community mit ungef"ahr 1500 aktiven Mitgliedern unterst"utzt.

\subsection{Lizenz}
Odoo ist mit der AGPL(Affero General Public License) versehen. AGPL ist eine Lizenz f"ur Freie Software mit Copyleft, die Nutzer m"ussen eine Downloadm"oglichkeit f"ur den Quelltext erhalten. Dadurch wird das sogenannte ASP-Schlupfloch der GPL geschlossen. Das ASP(Application Service Provider)-Schlupfloch erlaubt es Unternehmen, die eine GPL-Software im Netz betreiben, diese also nicht weitergeben, den Quelltext f"ur sich zu behalten. Da das nicht der Sinn dieser Lizenz ist, wird AGPL von der Free Software Foundation favorisiert.


\subsection{Kurzbeschreibung}
Wie bereits erw"ahnt und beschrieben ist Odoo ein ERP-System. Es ist geschrieben in Python und wird weltweit momentan von ungef"ahr 2 Millionen Unternehmen aller Gr"o"sen zum managen genutzt.
Die Hauptkomponenten sind ein Server, 260 Kernmodule(auch offizielle Module genannt) und ungef"ahr 4000 Community-Module.
Wie das Unternehmen wurde auch der Name des Produkts von OpenERP zu Odoo ge"andert. Dies hat den Grund, dass 2014 mit dem Erscheinen der Version 8 einige zus"atzliche Applikationen hinzugef"ugt wurden, welche man nicht zu einem ERP-System z"ahlen kann. Darunter waren Website-Builder, E-Commerce Software, PoS(Point of Sale) und Business Intelligence.
Die offiziellen Odoo Applikationen sind in sechs Gruppen eingeteilt:

\begin{itemize}
	\item Front-end apps: website builder, blog, e-commerce
	\item Sales management apps: CRM, point of sales, quotation builder
	\item Business operations apps: project management, inventory, manufacturing, accounting and purchase
	\item Marketing apps: mass mailing, lead automation, events, survey, forum, live chat
	\item Human Resources apps: employee directory, enterprise social network, leaves management, timesheet, fleet management
	\item Productivity apps: business intelligence, instant messaging, notes
\end{itemize}

Die neusten Versionen von Odoo (ab Version 7) sind gr"o"stenteils als Webapplikationen implementiert. Der Odoo Server bietet die Odoo Applikationen online an. Er speichert die Kundendaten "uber ein Interface in einer Datenbank. Es gibt auch den Webclient um den Zugang von Webbrowsern zu gew"ahrleisten. Der Server ist gr"o"stenteils in Python geschrieben, der Client in JavaScript.

\subsection{Verwendung aktuell}
Wie bereits erw"ahnt, wird Odoo weltweit von 2 Millionen Unternehmen jeglicher Gr"o"sen verwendet. Einige Beispiele sind Danone, Auchan, Veolia, La Poste, Singer.
