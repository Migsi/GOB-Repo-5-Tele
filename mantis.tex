section{Mantis}
Mantis ist eine freie Software, welche zur Auffindung von Bugs im eigenem
Projekt dient. Mithilfe von Mantis kann man Programmfehler Verfolgen und
Verwalten.
\subsection{Geschichte}
Kenzaburo IT fing im Jahre 2000 mit der Entwicklung des Programms an.
Im Jahre 2002 wurde Victor Boctor Projektleiter. Die erste Version (1.0.0)
wurde im Februar 2006 veroeffentlicht. Die darauffolgende Version 1.1.0 wur-
de dann im Dezember 2007 herausgegeben. Im November des Jahres 2008
wurde nach langen Diskussionen von Subversion Revision Control zu Git
gewechselt. Zwei Jahre spaeter kam die Version 1.2.0 heraus. Im Juli 2012
wurde die MantisBT Organisation auf GitHub zum offiziellen Repository
für den Source Code des Projektes.
\subsection{Funktionsweise}
Mantis basiert auf PHP und benotigt einen Webserver. Der Benutzer be-
dient Mantis ueber eine HTML-Oberflache. Eine Datenbank (MySQL, MS
SQL oder PostgreSQL) wird verwendet, um die Eintraege zu verwalten. Man-
tis ist unter Linux, Mac OS X, Windows, OS/2 und Unix lauffahig. Es ist
ueber die Konfigurationsdatei config inc.php weitgehend konfigurierbar. Be-
nutzerbezogene Einstellungen koennen ueber die Web-Oberflaeche vorgenom-
men werden. Um ueber externe Software Zugriff auf die Inhalte des MantisBT
zu haben, entwickelten sich im Laufe der Jahre verschiedene Schnittstellen,
welche den Zugriff ueber die, von Mantis bereitgestellten, SOAP-Webservices
vereinfachen. Einer der bekanntesten Vertreter ist das freie Projekt Man-
tisConnect, welches sowohl eine Java als auch eine .NET Bibliothek zur
Verfuegung stellt.
\subsection{Methodik}
In Mantis koennen verschiedene Projekte angelegt werden. Auch eine Unter-
gliederung in Unterprojekte ist moeglich. Den Projekten werden Projektteil-
nehmer mit unterschiedlichen Zugriffsrechten zugeteilt. Die Zugriffsrechte
sind ebenenbasiert: hoehere Zugriffsebenen (z. B. Entwickler) schließen die
Rechte niedrigerer Ebenen (z. B. Reporter) ein. Insgesamt gibt es sechs
vorgegebene Zugriffsebenen (Betrachter, Reporter, Tester, Entwickler, Ma-
nager, Administrator). Hat jemand mindestens Reporter-Status innerhalb
eines Projekts, kann er einen Problembericht (Issue) anlegen. Gegebenen-
falls kann dieser Bericht sofort einem Bearbeiter (= Projektteilnehmer mit
mindestens Entwickler-Status) zugeordnet werden. Jeder Problembericht be-
findet sich in einem von mehreren vom eingebauten Arbeitsablauf vorgege-
1benen Zustaenden (z. B. Neu, Zugewiesen, Behoben, Geschlossen). Fuer Zu-
standsaenderungen bedarf es wiederum entsprechender Zugriffsrechte. Waehrend
des Lebenszyklus eines Fehlerberichts koennen von allen berechtigten Pro-
jektteilnehmern zu jedem Zeitpunkt Kommentare zum Bericht hinzugefuegt
werden. Das System bietet bei Zustandswechseln ebenfalls eine Kommentar-
funktion an, sodass der Lebenszyklus eines Berichts nachvollzogen werden
kann.



