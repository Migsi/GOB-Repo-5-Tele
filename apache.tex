
\section{Apache Software Foundation}
Die Apache Software Foundation ist eine non-Profit Organisation,eingetragen in den Vereinigten Staaten von Amerika, welche an viele verschieden Projekte f"ur die Gemeinn"utzigkeit arbeitet. Die Foundation wurde 1911 gegr"undet und hat seid dem die folgenden Aufgaben:
\begin{itemize}
\item{Eine Grundlage f"ur offene, kollaborative Software-Entwicklungsprojekte zu schaffen und diese mit Hardware,Kommunikation und Infrastrukturen zu versorgen. }
\item{Eine eigenen st"andige juristische Entit"at zu schaffen, in der Unternehmen und Einzelpersonen spenden k"onnen,und sich sicher sein k"onnen, dass diese Ressourcen f"ur das Gemeinwohl eingesetzt werden.}
\item{die 'Apache' Marke zu schützen}
\end{itemize}
Im Gegensatz zu andere Opensource-Projekte wurde der Apache Webserver nicht nur von einem Entwickler gestartet(zum Beispiel bei Linux der Kernel),sondern begann als eine vielf"altige Gruppe,die gemeinsame Interessen teilten und sich durch den Austausch von  von Informationen, Korrekturen und Anregungen kennen lernten.
Als die Gruppe begann, ihre eigene Version der Software zu entwickeln, wurden immer mehr Menschen angezogen und begannen zu helfen  zuerst  mit Anregungen oder kleine Patches teilten, sp"ater dann auch mit wichtigeren Beitr"agen. \\
Apache nannte dieses Grundprinzip ``Leistungsgesellschaft'': wortw"ortlich Regierung durch Verdienst.\\
Als der Apache Server gr"o"ser und gr"o"ser  wurde aufgrund der offenen Gemeinheit hinter dem Projekt, begannen die Menschen an Satelliten-Projekte  zu arbeiteten  und verbreiteten die selber Prinzipien auch in denen. Als dann die Apache Software Fondation gegr"undet worden ist gab es eine sehr gro"se Anzahl an verschiedenen Projekten. Um die Vielfalt zu bewahren entschloss man sich das Gesamte nicht Zentral zu regeln, sondern jedes Projekt soll die Autorit"at "uber die Entwicklung seiner Software sein und hat so ein hohes Ma"s an Freiheit bei der Gestaltung ihrer eigenen Regeln. \\
\newpage 
Die Fondation bekam trotzdem ihre Leitung. Folgende Entit"aten wurden erschafft:
\begin{itemize}
\item{Der Verwaltungsrat (Board of Directors ) leitet die Foundation und besteht aus Mitgliedern}
\item{Die Projekt Management Committees (PMC) , diese regeln die Projekte}
\end{itemize}

Apache st"utzt sich gro"steils auf  freiwillige Helfer und Entwickler, die Community spielt daher eine wesentliche Rolle in der Planung, Umsetzung und Realisierung der Projekte, in der Community gibt es verschieden ´´Rollen´´ von Benutzer:
\begin{itemize}
\item{Die \textbf{USER}: Ein Benutzer ist jemand, der Apache Software verwendet. Benutzer tragen durch R"uckmeldungen an die Entwickler in Form von Bug-Reports und Feature-Vorschl"age zu den Projekten bei. Nutzer nehmen an der Apache-Community, indem sie andere neue Benutzer helfen teil.}
\item{Die \textbf{DEVELOPER}: Ein Entwickler ist ein Benutzer, welcher in Form von Code oder Dokumentation zu einem Projekt beitr"agt. Entwickler nehmen an Diskussionen teil, bieten Patches, Vorschl"age und Kritik. Entwickler sind auch als  ´´contributors´´ bekannt.}
\item{Die \textbf{COMMITER}:Ein Committer ist ein Entwickler, der Schreibzugriff auf die Code-Repository hat.Commiter haben eine apache.org Mail-Adresse. Dadurch das sie keine weiteren Personen für die Patches brauchen, treffen Commiter meistens kurzfristige Entscheidungen f"ur ein Projekt. Das Projekt Management Commitee kann diese Entscheidungen genehmigen sodass sie permanent bleibt oder sie ablehnen. Die PMC treffen die Entscheidungen nicht die Committer.}
\item{\textbf{PMC Member}:Ein PMC Member ist ein Entwickler oder ein Committer, der aufgrund seiner au"serordentliche Verdienste f"ur die Entwicklung eines Projekts oder aufgrund seines au"serordentliches  Engagement im Comitee gewählt worden ist. Er hat alle Rechte die Ein Commiter besitzt zudem kann er an Community-verbundene Entscheidungen mitbestimmen und ein aktiven Benutzter als Commiter vorschlagen. Das PCM als ganzes regelt das Projekt }
\end{itemize}
\newpage
Die Community von Apache folgt den sechs Prinzipien die auch ´´The Apache Way´´ bezeichnet werden:
\begin{itemize}
\item{kollaborative Software-Entwicklung}
\item{Handelsfreundliche Standard-Lizenz}
\item{Konstante Software Qualität}
\item{Respektvoll, ehrlich, technische basierte Interaktion}
\item{Umsetzung von Standards}
\item{Sicherheit als Pflicht Funktion}
\end{itemize}
Projekte Regeln sich normalerweise von selbst, von freiwilligen Nutzer. Macht deren die tun,dies funktioniert gut f"ur die meisten F"alle. Wenn Koordinierung erforderlich ist, werden Beschl"usse durch einer Abstimmung beschlossen.

