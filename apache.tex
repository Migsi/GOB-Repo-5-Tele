



Die Apache Software Foundation ist eine non-Profit Organisation,eingetragen in den Vereinigten Staaten von Amerika, welche an viele verschieden Projekte für die Gemeinnützigkeit arbeitet. Die Foundation wurde 1911 gegründet und hat seid dem die folgenden Aufgaben:
\begin{itemize}
\item{Eine Grundlage für offene, kollaborative Software-Entwicklungsprojekte zu schaffen und diese mit Hardware,Kommunikation und Infrastrukturen zu versorgen. }
\item{Eine eigenen ständige juristische Entität zu schaffen, in der Unternehmen und Einzelpersonen spenden können,und sich sicher sein können, dass diese Ressourcen für das Gemeinwohl eingesetzt werden.}
\item{die 'Apache' Marke zu schützen}
\end{itemize}
Im Gegensatz zu andere Opensource-Projekte wurde der Apache Webserver nicht nur von einem Entwickler gestartet(zum Beispiel bei Linux der Kernel),sondern begann als eine vielfältige Gruppe,die gemeinsame Interessen teilten und sich durch den Austausch von  von Informationen, Korrekturen und Anregungen kennen lernten.
Als die Gruppe begann, ihre eigene Version der Software zu entwickeln, wurden immer mehr Menschen angezogen und begannen zu helfen indem sie zuerst Anregungen oder kleine Patches teilten, später dann mit wichtigeren Beiträgen. \\
Apache nannte dieses Grundprinzip "Leistungsgesellschaft": wortwörtlich Regierung durch Verdienst.\\
Als der Apache Server größer und größer wurde aufgrund der offenen Gemeinheit hinter dem Projekt, begannen die Menschen Satelliten-Projekte an oder arbeiteten an denen und verbreiteten die selber Prinzipien auch in denen. Als dann die Apache Software Fondation gegründet worden ist gab es eine sehr große Anzahl an verschiedenen Projekten. Um die Vielfalt zu bewahren entschloss man sich das gesamte nicht Zentral zu regeln, sondern jedes Projekt ist die Autorität über die Entwicklung seiner Software und hat so ein hohes Maß an Freiheit bei der Gestaltung ihrer eigenen Regeln. \\
 \newpage 
Die Fondation bekam trotzdem ihre Leitung. Folgende Entitäten wurden erschafft:
\begin{itemize}
\item{Der Verwaltungsrat (Board of Directors ) leitet die Foundation und besteht aus Mitgliedern}
\item{Die Projekt Management Committees (PMC) , diese regeln die Projekte}
\end{itemize}

Apache stützt sich großteils auf  freiwillige Helfer und Entwickler, die Community spielt daher eine wesentliche Rolle in der Planung, Umsetzung und Realisierung der Projekte, in der Community gibt es verschieden "Rollen" von Benutzer:
\begin{itemize}
\item{Die \textbf{USER}: Ein Benutzer ist jemand, der Apache Software verwendet. Benutzer tragen durch Rückmeldungen an die Entwickler in Form von Bug-Reports und Feature-Vorschläge zu den Projekten bei. Nutzer nehmen an der Apache-Community, indem andere Benutzer helfen teil.}
\item{Die \textbf{DEVELOPER}: Ein Entwickler ist ein Benutzer, welcher in Form von Code oder Dokumentation zu einem Projekt beiträgt. Entwickler nehmen an Diskussionen teil, bieten Patches, Vorschläge und Kritik. Entwickler sind auch als  "contributors" bekannt.}
\item{Die \textbf{COMMITER}:Ein Committer ist ein Entwickler, der Schreibzugriff auf die Code-Repository hat.Commiter haben eine apache.org Mail-Adresse. Dadurch das sie keine weiteren Personen für die Patches brauchen, treffen Commiter meistens kurzfristige Entscheidungen für ein Projekt. Das Projekt Management Commitee kann diese Entscheidungen genehmigen sodass sie permanent bleibt oder sie ablehnen. Die PMC treffen die Entscheidungen nicht die Committer.}
\item{\textbf{PMC Member}:Ein PMC Member ist ein Entwickler oder ein Committer, der aufgrund seiner außerordentliche Verdienste für die Entwicklung eines Projekts oder aufgrund seines außerordentliches  Engagement im Comitee gewählt worden ist. Er hat alle Rechte die Ein Commiter besitzt zudem kann er an Community-verbundene Entscheidungen mitbestimmen und ein aktiven Benutzter als Commiter vorschlagen. Das PCM als ganzes regelt das Projekt }
\end{itemize}
\newpage
Die Community von Apache folgt den sechs Prinzipien die auch "The Apache Way" bezeichnet werden:
\begin{itemize}
\item{kollaborative Software-Entwicklung}
\item{Handelsfreundliche Standard-Lizenz}
\item{Konstante Software Qualität}
\item{Respektvoll, ehrlich, technische basierte Interaktion}
\item{Umsetzung von Standards}
\item{Sicherheit als Pflicht Funktion}
\end{itemize}
Projekte Regeln sich normalerweise von selbst, von freiwilligen Nutzer. Macht deren die tun,dies funktioniert gut für die meisten Fälle. Wenn Koordinierung erforderlich ist, werden Beschlüsse durch einer Abstimmung beschlossen.

