
\begin{center}
\textbf{\LARGE{Apache}}
\end{center}


\section{Einleitung}
Apache ist ein Open-Source HTTP Server und freies Produkt der Apache Software Foundation, er ist heutzutage der meist benutzte HTTP Server  im Internet.

\section{Geschichte}
Apache wurde 1995 von einer Gruppe von acht Entwickler veröffentlicht, diese hatten als Aufgabe die Weiterentwicklung des  Webservers NCSA HTTPd. Die bedeutendsten dieser Entwickler waren Brian Behlendorf und Roy T. Fielding.
Der Name Apache stammt aus dem Nordamerikanische Indianerstamm Apachen. Eine falsche Interpretation ist,dass der Name eine Umdeutung von „a patchy server“ sei, was so viel wie zusammengeflickter Server heißt, diese Deutung entstand dadurch das der Apache eine gepatchte erweiterung des alten Webservers NCSA HTTPd.

\section{Distributionen}
Der Apache Server ist standardmäßig auf jeder Linux-Distribution enthalten, für Windows kann man die Entwicklungs-Distribution XAMPP installieren. Die für den Apache-Webserver genutzte Architektur unterscheidet sich in der 
Version 1.3 stark zwischen den Unix-Versionen und der Version des Apache-
Webservers für die Windows-Plattform.

\section{Erweiterungen}
Der Apache Server kann mit sogenannte "Module" erweitert werden um benutzerspezifische
Anforderungen zu erfüllen die bekanntesten module sind: 
\begin{itemize}
\item{SSL}
\item{Einbindung von PHP5 oder python}
\item{Weiterleitung an Proxy-Servern}
\item{Automatische Erzeugung von Statusberichten}
\end{itemize}



\newpage



\section{Funktionsweiße}
Um die Funktionsweise des Apache-Webservers zu veranschaulichen, bietet es 
sich an, die Abarbeitung in verschiedene Stufen zu unterteilen.
\begin{description}

\item{Übersetzung der URL in einen Dateinamen\\  Hier wird aus der vom 
Webclient angeforderten URL der entsprechende Dateiname im Filesystem 
des Webservers generiert.}

\item{Überprüfung der Authentisierung \\ Hier findet die Überprüfung eventuell 
vom Benutzer übergebener Datei beziehungsweise Datenbank statt}

\item{Prüfung der Zugriffsberechtigungen \\Hier wird überprüft, ob ausreichende 
Zugriffsrechte für den Zugriff auf die angeforderte Ressource durch den jeweiligen Client bestehen}

\item{Bestimmung des MIME-Typs des jeweiligen Dateinamen \\IME-Typ bestimmt unter anderem auch die Art und Weise, wie der Apache-Webserver einen Request weiter 
bearbeitet.  }

\item{Erstellen und Senden der Antwort an den Webclient \\ das Erstellen der 
Antwort an den Webclient hängt von einer Vielzahl von Faktoren 
ab. Zu nennen wären dabei die Art des HTTP-Requests, der MIME-Typ, der Datei, auf die die URL abgebildet wird, die Konfiguration des Apache-Webservers und anderes mehr.}


\item{Loggen des Requests \\ Die Bearbeitung eingehender Requests geschieht anhand des sogenannten Handlers. Ein Handler ist die Darstellung der Vorgänge die zur Erstellung der Antwort auf einen Request abgearbeitet werden. }
\end{description}
Die Standard-Distribution des Apache-Webservers enthält mehrere Handler unter anderem dem default-Handler. Allerdings gibt es die Möglichkeit für den Benutzer eigene Handler und so mit eigene Vorgänge zu definieren.

\newpage


\section{Sicherheitsaspekte}

\subsection{Konfiguration}
Die genaue Konfiguration des Apache Servers ist ein wesentlicher Punkt der zur Sicherheit des Systems beitragen kann. Der Apache Server ließt aus drei Dateien:
\begin{itemize}
\item{httpd.conf}
\item{srm.conf}
\item{access.conf}
\end{itemize}
Einige der grundlegensten konfigurationen sind:
\begin{description}
\item{ServerName \\ Hostname des Webservers}
\item{DocumentRoot \\ Wurzelverzeichnis der auf diesem Webserver abgelegten 
Dokumente }
\item{Deny,Allow \\ Diese Verweigern oder Erlauben den Zugriff entsprechend der IP-Adresse beziehungsweise des Hostnames }
\end{description} 

\subsection{Bekannte Angriffe und Schwachstellen}
Der bekannteste Angriff auf den Apache-Server ist so wie auf jedem anderen HTTP-Server der DOS-Angriff, DOS bedeutet soviel wie Denial-of-Service, dabei wird der Server durch der Anzahl der Anfragen, die von einem Angreifer kommen ausgelastet,sodass er nicht mehr auf den anderen Anfragen reagieren kann. Der Server antwortet dann mit dem DOS beziehungsweise mit dem Denial-of-Service Dos-Attacken kann man mit verschiedenen Firewalls regeln unterbinden. Ein Ähnlicher Angriff ist DDOS das bedeutet soviel wie Distributed Denial of Service, zum Unterschied zu DOS greift bei DDOS der Angreifer nicht nur mit einer Rechner den Server an sondern mit hunderten von Rechner die alle eine andere IP haben an. DDOS-Attacken kann man nicht unterbinden.
