\subsection*{DokuWiki}
DokuWiki ist eine freie, plattformunabh"angige Wiki-Software, die in der Programmiersprache PHP realisiert wurde. Sowohl die Inhalte, als auch die Metadaten werden in Textdateien gespeichert und zu Gunsten der Leserlichkeit strikt voneinander getrennt. Wiki-Software wird zur gemeinschaftlichen Erstellung, Bearbeitung und Organisation von sogenannten Wikis verwendet. Ein Wiki ist ein Hypertextsystem f"ur Webseiten, bei denen Benutzer Texte lesen und direkt im Browser ver"andern k"onnen. Es dient zum gemeinschaftlichen Sammeln von Wissen. DokuWiki wurde als Hilfsmittel zur einfachen Dokumentation von Projekten gedacht, wird allerdings heutzutage wegen seiner Einfachheit und Funktionen f"ur eine Vielzahl von Anwendungen eingesetzt.
\subsubsection*{Geschichte}
Die erste Version von DokuWiki wurde im Juli 2004 von Andreas Gohr auf Freshmeat ver"offentlicht. Im Januar 2005 folgte mit der "Uberarbeitung des Parsers und des Renderers ein Meilenstein in der Geschichte des Projekts. Dadurch konnte die Software nun auch f"ur gr"oßere Projekte verwendet und Add-Ons einfacher integriert werden. 2005 wurde DokuWiki in die Linuxdistributionen Debian und Gentoo Linux aufgenommen, was eine verst"arkte Verbreitung zur Folge hatte. Die aktuelle Version ist 2014-09-29a ?Hrun?, die am 8. Oktober 2014 ver"offentlicht wurde.
\subsubsection*{Lizenz}
DokuWiki wurde unter der GPL 2 Lizenz lizenziert, was bedeutet, dass man die Software kostenlos nutzen, studieren, "andern und verbreiten darf. Auf der Website wird lediglich darauf hingewiesen, dass sich mit Spenden bedanken kann, falls man die Software in seinem Unternehmen nutzt.
\subsubsection*{Funktionen}
\begin{itemize}
\item \textbf{Versionsverwaltung}
Alle Versionen einer Wikiseite werden gespeichert und k"onnen miteinander verglichen werden. Die gleichzeitige Ver"anderung einer Seite durch mehrere Benutzer wird verhindert.
\item \textbf{Zugriffskontrolle}
Es k"onnen Zugriffsrechte f"ur Kombinationen von Benutzern, Gruppen und Namespaces vergeben werden.
\item \textbf{Add-Ons}
Neben der M"oglichkeit, selbst Erweiterungen in PHP zu schreiben, gibt es zahlreiche Add-Ons, die mittels eines Plug-In-Managers integriert und verwaltet werden k"onnen.
\item \textbf{Templates}
Das Aussehen der Wikis kann "uber unterschiedliche Templates ver"andert werden.
\item \textbf{Internationalisierung}
Durch die Verwendung von UTF-8 ist das Wiki in 39 Sprachen verf"ugbar.
\item \textbf{Caching}
Ein Cache speichert geparste Seiten und macht somit ein erneutes Parsen einer bereits aufgerufenen Seite "uberfl"ussig.
\item \textbf{Volltextsuche}
"Uber die Volltextsuche kann nach Stichw"ortern gesucht werden.
\item \textbf{Portable Version}
Es gibt eine portable Version f"ur Windows, um das Programm "uber einen USB-Stick laufen zu lassen.
\item \textbf{Einfache Integration}
Durch die reine Nutzung von .txt Dateien ist keine Datenbank erforderlich.
\item \textbf{Hohe Bedienfreundlichkeit}
DokuWiki bietet Funktionen, wie Rechtschreibpr"ufung und automatisches Erstellen von Inhaltsverzeichnissen zur Verf"ugung. 
\end{itemize}

\subsubsection*{Verbreitung}
Laut einer Anfang 2012 vom t3n Open.Web.Business Magazin durchgef"uhrten Bewertung liegt DokuWiki auf Platz drei der Open Source Wikis. Eine starke Konkurrenz stellen Confluence und MediaWiki dar, die als Platzhirsche auf dem Gebiet gelten.
