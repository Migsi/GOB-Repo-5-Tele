\subsection*{DokuWiki}
DokuWiki ist eine freie, plattformunabhängige Wiki-Software, die in der Programmiersprache PHP realisiert wurde. Sowohl die Inhalte, als auch die Metadaten werden in Textdateien gespeichert und zu Gunsten der Leserlichkeit strikt voneinander getrennt. Wiki-Software wird zur gemeinschaftlichen Erstellung, Bearbeitung und Organisation von sogenannten Wikis verwendet. Ein Wiki ist ein Hypertextsystem für Webseiten, bei denen Benutzer Texte lesen und direkt im Browser verändern können. Es dient zum gemeinschaftlichen Sammeln von Wissen. DokuWiki wurde als Hilfsmittel zur einfachen Dokumentation von Projekten gedacht, wird allerdings heutzutage wegen seiner Einfachheit und Funktionen für eine Vielzahl von Anwendungen eingesetzt.
\subsubsection*{Geschichte}
Die erste Version von DokuWiki wurde im Juli 2004 von Andreas Gohr auf Freshmeat veröffentlicht. Im Januar 2005 folgte mit der Überarbeitung des Parsers und des Renderers ein Meilenstein in der Geschichte des Projekts. Dadurch konnte die Software nun auch für größere Projekte verwendet und Add-Ons einfacher integriert werden. 2005 wurde DokuWiki in die Linuxdistributionen Debian und Gentoo Linux aufgenommen, was eine verstärkte Verbreitung zur Folge hatte. Die aktuelle Version ist 2014-09-29a ?Hrun?, die am 8. Oktober 2014 veröffentlicht wurde.
\subsubsection*{Lizenz}
DokuWiki wurde unter der GPL 2 Lizenz lizenziert, was bedeutet, dass man die Software kostenlos nutzen, studieren, ändern und verbreiten darf. Auf der Website wird lediglich darauf hingewiesen, dass sich mit Spenden bedanken kann, falls man die Software in seinem Unternehmen nutzt.
\subsubsection*{Funktionen}
\begin{itemize}
\item \textbf{Versionsverwaltung}
\\
Alle Versionen einer Wikiseite werden gespeichert und können miteinander verglichen werden. Die gleichzeitige Veränderung einer Seite durch mehrere Benutzer wird verhindert.
\item \textbf{Zugriffskontrolle}
\\
Es können Zugriffsrechte für Kombinationen von Benutzern, Gruppen und Namespaces vergeben werden.
\item \textbf{Add-Ons}
\\
Neben der Möglichkeit, selbst Erweiterungen in PHP zu schreiben, gibt es zahlreiche Add-Ons, die mittels eines Plug-In-Managers integriert und verwaltet werden können.
\item \textbf{Templates}
\\
Das Aussehen der Wikis kann über unterschiedliche Templates verändert werden.
\item \textbf{Internationalisierung}
\\
Durch die Verwendung von UTF-8 ist das Wiki in 39 Sprachen verfügbar.
\item \textbf{Caching}
\\
Ein Cache speichert geparste Seiten und macht somit ein erneutes Parsen einer bereits aufgerufenen Seite überflüssig.
\item \textbf{Volltextsuche}
\\
Über die Volltextsuche kann nach Stichwörtern gesucht werden.
\item \textbf{Portable Version}
\\
Es gibt eine portable Version für Windows, um das Programm über einen USB-Stick laufen zu lassen.
\item \textbf{Einfache Integration}
\\
Durch die reine Nutzung von .txt Dateien ist keine Datenbank erforderlich.
\item \textbf{Hohe Bedienfreundlichkeit}
\\
DokuWiki bietet Funktionen, wie Rechtschreibprüfung und automatisches Erstellen von Inhaltsverzeichnissen zur Verfügung.
\end{itemize}

\subsubsection*{Verbreitung}
Laut einer Anfang 2012 vom t3n Open.Web.Business Magazin durchgeführten Bewertung liegt DokuWiki auf Platz drei der Open Source Wikis. Eine starke Konkurrenz stellen Confluence und MediaWiki dar, die als Platzhirsche auf dem Gebiet gelten.
