\section{Sourceforge}
SourceForge (englisch f"ur ''Quelltextschmiede'') ist ein Repository in Form einer Website,
welche Programmierern die M"oglichkeit bietet, quelloffene Softwareprojekte zu erstellen und zu verwalten. 
Es basiert auf dem gleichnamigen Softwareentwicklungssystem und wird vom US-amerikanischen Unternehmen 
Dice Holdings betrieben.
\subsection{Software}
SourceForge wird von Dice Holdings (bis 2012 GeekNet) entwickelt und zur Verf"ugung gestellt.
Sie stellt eine Plattform dar, die eine breite Dienstleistungspalette anbietet. Dazu geh"oren zahlreiche Open-Source-Anwendungen
wie GNU Mailman oder Trac, sowie diverse Versionsverwaltungen f"ur Quelltext (z.B. Git, Mercurial, SVN...).
Seit Version 3 ist SourceForge nicht mehr als freie Software erh"altlich. Jedoch arbeitet ein SourceForge-Programmierer
parallel zur offiziellen Plattform an einer Open-Source-Variante (GForge). Nachdem aus dieser Variante 
abermals eine propriet"are Software mit dem Namen GForge AS entstand, wurde die freie Variante unter dem Namen FusionForge
weitergef"uhrt. Als Alternative brachte die Free Software Foundation Savannah heraus. Dieses basiert auf
auf der Version 2 der SourceForge-Software.
Seit April 2013 werden die auf Sourceforge liegenden Projekte auf Allura migriert.
\subsection{Portal}
SourceForge fungiert als Plattform zur Entwicklung von Open-Source-Anwendungen. Durch die breite Basis an
Dienstleistungen, wird es von vielen Entwicklern zur Verwaltung von Softwareprojekten genutzt. Daf"ur zum Einsatz kommen
hier beispielsweise Git, SVN, CVS oder andere Derivate. Zudem k"onnen f"ur jedes Projekt
ein eigenes Wiki, sowie zugeh"orige Datenbanken angelegt werden. Im Mai 2013 z"ahlte SourceForge
mehr als 300.000 Projekte und "uber 3 Millionen registrierte Nutzer.
Mehrere große Projekte werden von SourceForge gehostet. Dazu geh"ort z.B. der Torrent-Multimedia Client Vuze,
der DOS Emulator ''DOSBox'' und der Datenbankeditor phpMyAdmin.
\subsection{Kritik}
Seit August 2013 bietet SourceForge sogenannte Drive-by-Installer an, die bei der Installation 
neben der gew"unschten Software auch Adware vorschl"agt. Das sorgt f"ur einigen Unmut in der Community. 
Unter den verbreiteten Programmen befinden sich z.B. die Ask-Toolbar. Die Verwendung des
Drive-by-Installers ist jedoch optional und muss vom Entwickler explizit aktiviert werden (opt-in).

