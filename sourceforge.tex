\section{Sourceforge}
SourceForge (englisch für ``Quelltextschmiede'') ist ein Repository in Form einer Website,
welche Programmierern die Möglichkeit bietet, quelloffene Softwareprojekte zu erstellen und zu verwalten. 
Es basiert auf dem gleichnamigen Softwareentwicklungssystem und wird vom US-amerikanischen Unternehmen 
Dice Holdings betrieben.
\subsection{Software}
SourceForge wird von Dice Holdings (bis 2012 GeekNet) entwickelt und zur Verfügung gestellt.
Sie stellt eine Plattform dar, die eine breite Dienstleistungspalette anbietet. Dazu gehören zahlreiche Open-Source-Anwendungen
wie GNU Mailman oder Trac, sowie diverse Versionsverwaltungen für Quelltext (z.B. Git, Mercurial, SVN...).
Seit Version 3 ist SourceForge nicht mehr als freie Software erhältlich. Jedoch arbeitet ein SourceForge-Programmierer
parallel zur offiziellen Plattform an einer Open-Source-Variante (GForge). Nachdem aus dieser Variante 
abermals eine proprietäre Software mit dem Namen GForge AS entstand, wurde die freie Variante unter dem Namen FusionForge
weitergeführt. Als Alternative brachte die Free Software Foundation Savannah heraus. Dieses basiert auf
auf der Version 2 der SourceForge-Software.
Seit April 2013 werden die auf Sourceforge liegenden Projekte auf Allura migriert.
\subsection{Portal}
SourceForge fungiert als Plattform zur Entwicklung von Open-Source-Anwendungen. Durch die breite Basis an
Dienstleistungen, wird es von vielen Entwicklern zur Verwaltung von Softwareprojekten genutzt. Dafür zum Einsatz kommen
hier beispielsweise Git, SVN, CVS oder andere Derivate. Zudem können für jedes Projekt
ein eigenes Wiki, sowie zugehörige Datenbanken angelegt werden.
Mehrere große Projekte werden von SourceForge gehostet. Dazu gehört z.B. der Torrent-Multimedia Client Vuze.
\subsection{Kritik}
Seit August 2013 bietet SourceForge sogenannte Drive-by-Installer an, die bei der Installation 
neben der gewünschten Software auch Adware vorschlägt. Das sorgt für einigen Unmut in der Community. 
Unter den verbreiteten Programmen befinden sich z.B. die Ask-Toolbar. Die Verwendung des
Drive-by-Installers ist jedoch optional und muss vom Entwickler explizit aktiviert werden (opt-in).

