\section{Sourceforge}
SourceForge (englisch für Quelltextschmiede) ist ein Repository in Form einer Website, 
welche Programmierern die Möglichkeit bietet, quelloffene Softwareprojekte zu erstellen und zu verwalten. 
Es basiert auf dem gleichnamigen Softwareentwicklungssystem und wird vom US-amerikanischen Unternehmen 
Dice Holdings betrieben.
\subsection{Software}
Die Software SourceForge wird von Dice Holdings (bis 2012 GeekNet) entwickelt und vertrieben. 
Sie stellt ein Portal zu einer Dienstleistungspalette zur Verfügung und integriert eine 
Anzahl von Open-Source-Anwendungen (z. B. GNU Mailman, Trac). Als Versionsverwaltung für Quelltext (SCM) 
werden CVS, SVN, Bazaar, Git und Mercurial angeboten. Bis Version 3 war SourceForge als freie 
Software verfügbar, wurde danach jedoch kommerziell und proprietär vertrieben von VA Software. 
Im Jahr 2007 verkaufte VA Software die SourceForge Enterprise Edition an die kalifornische 
CollabNet Inc.[2] Außerdem entwickelt ein SourceForge-Programmierer die Software aber auch unter dem 
Namen GForge als Open-Source-Projekt weiter. Unter dem Namen GForge AS entstand abermals eine proprietäre 
Software mit dem Namen. Um der Verwirrung um den Namen zu entgehen, wird die freie Version des 
ursprünglichen GForge als FusionForge weitergeführt. Die Free Software Foundation stellte mit Savannah eine 
Alternative für die proprietäre SourceForge-Software, welches auf GNU Savannah verwendet wird. 
Savannah basiert auf der Version 2 der SourceForge-Software. Im Juni 2012 wurde vorgeschlagen, 
die Neuimplementierung der SourceForge-Software namens Allura, dem Apache-Projekt zu übergeben. 
Seit dem 22. April 2013 werden die bei Sourceforge liegenden Projekte auf Allura migriert.
\subsection{Portal}
Das Webportal SourceForge.net dient zur Entwicklung von Open-Source-Programmen und wird von vielen 
Software-Entwicklern zur Verwaltung ihrer Projekte genutzt. Die Website wird von SourceForge, Inc. 
gehostet und nutzt die SourceForge-Software. Die Software bietet unterschiedliche Systeme für 
die Versionsverwaltung, wie etwa CVS, SVN, Bazaar oder Git. Des Weiteren kann jedes Projekt ein 
eigenes Wiki anlegen und es kann auf eine eigene MySQL-Datenbank zugegriffen werden. 
Viele große Open-Source-Projekte werden von SourceForge gehostet, aber es gibt auch kleine oder 
inaktive Projekte. Zu den größten Projekten gehören unter Anderem eMule, Vuze und Ares Galaxy mit 
jeweils mehreren hundert Millionen Downloads.
\subsection{Sperrungen des Zugriffs}
Die chinesische Regierung hatte den Zugriff auf die Seite im Zuge vom Projekt Goldener Schild gesperrt, 
doch die Sperre wurde im darauffolgenden Jahr wieder aufgehoben.[6] Im Juni 2008 war das Portal von China 
aus erneut nicht erreichbar und es wurde über Zusammenhänge mit einem Programmierer von SourceForge, 
welcher die Chinesische Regierung negativ kritisierte, spekuliert. 
Da die US-Regierung seit längerer Zeit Handelsverbote und Sanktionen gegen so genannte Schurkenstaaten 
verhängt, verkündete SourceForge im Januar 2010 die Benutzer aus jenen Ländern, die auf der 
Sanktionsliste des US-amerikanischen Außenministeriums gelistet sind, von der Benutzung auszuschließen. 
Im Januar 2010 war die Seite somit im Iran, in Kuba, Syrien, Nordkorea und im Sudan nicht 
abrufbar. Auf Grund teils heftiger Reaktionen der Community und dem Bekenntnis von SourceForge zu 
dem Open-Source-Prinzip des freien Informationsaustauschs [9] gab Sourceforge am 7. Februar 2010 bekannt, 
dass diese Länder nicht mehr generell vom Download der Software ausgeschlossen sind, sondern 
jeder Projektadministrator diese Einschränkung selber für sein Projekt vornehmen kann.
\subsection{Kritik}
Seit August 2013 bietet SourceForge sog. Drive-by-Installer an, die bei der Installation 
außer der gewünschten Software auch Adware von Drittanbietern zur Installation vorschlägt. 
Unter den verbreiteten Programmen befinden sich u. a. die Ask-Toolbar und das VPN-Tool HotspotShield. 
Diese Programme sind werbefinanziert und zeigen beim Surfen im Internet permanent Werbebanner bzw. 
sammeln Nutzerdaten.[11] Die Verwendung des Drive-by-Installers ist optional und muss vom Entwickler 
explizit aktiviert werden (opt-in).

